This code is a testing platform for the Common Community Physics Package (C\+C\+PP) and its driver. As of v0.\+1.\+0, this code simply calls a \char`\"{}do-\/nothing\char`\"{} physics suite with \char`\"{}do-\/nothing\char`\"{} physics schemes through the C\+C\+PP driver within the S\+CM framework. It represents a proof-\/of-\/concept and a platform on which to add proper physics in the coming months.

The C\+C\+PP and its driver are separated into its \href{https://github.com/NCAR/gmtb-ccpp}{\tt own repository} and are pulled in to this S\+CM code as a git submodule. The instructions below explain how to obtain the S\+CM code, bring in the C\+C\+PP as a submodule, build the combined code into an application, and run a simple case. Using the dummy physics, when the S\+CM runs, it performs all functions as if it is running a real case (including time-\/stepping), but instead of calling real physics, it calls the dummy physics, which simply print a confirmation message when called.

\subsection*{Obtaining Code}


\begin{DoxyEnumerate}
\item Download a compressed file or clone the source using
\end{DoxyEnumerate}
\begin{DoxyItemize}
\item {\ttfamily git clone \href{https://}{\tt https\+://}\mbox{[}username\mbox{]}@github.\+com/\+N\+C\+A\+R/gmtb-\/scm.git} and enter your github password when prompted.
\end{DoxyItemize}
\begin{DoxyEnumerate}
\item Change directory into the project.
\end{DoxyEnumerate}
\begin{DoxyItemize}
\item {\ttfamily cd gmtb-\/scm}
\end{DoxyItemize}
\begin{DoxyEnumerate}
\item Initialize the C\+C\+PP submodule.
\end{DoxyEnumerate}
\begin{DoxyItemize}
\item {\ttfamily git submodule init}
\end{DoxyItemize}
\begin{DoxyEnumerate}
\item Update the C\+C\+PP submodule.
\end{DoxyEnumerate}
\begin{DoxyItemize}
\item {\ttfamily git submodule update} and enter your github credentials again. If the machine is running an older version of git and you are denied access, you may need to configure the C\+C\+PP submodule U\+RL before repeating step 4 by executing this command\+:
\item {\ttfamily git config submodule.\+src/ccpp.url \href{https://}{\tt https\+://}\mbox{[}username\mbox{]}@github.\+com/\+N\+C\+A\+R/gmtb-\/ccpp.git}
\end{DoxyItemize}
\begin{DoxyEnumerate}
\item Change directory into the C\+C\+PP source.
\end{DoxyEnumerate}
\begin{DoxyItemize}
\item {\ttfamily cd src/ccpp}
\end{DoxyItemize}
\begin{DoxyEnumerate}
\item Checkout a branch or tag.
\end{DoxyEnumerate}
\begin{DoxyItemize}
\item {\ttfamily git checkout master} or
\item {\ttfamily git checkout v0.\+1.\+0}
\end{DoxyItemize}

\subsection*{Building and Compiling the S\+CM with C\+C\+PP}


\begin{DoxyEnumerate}
\item Change directory to the top-\/level directory.
\end{DoxyEnumerate}
\begin{DoxyItemize}
\item {\ttfamily cd ../..}
\end{DoxyItemize}
\begin{DoxyEnumerate}
\item \mbox{[}Optional\mbox{]} Run the machine setup script if necessary. This script loads compiler modules (Fortran 2003-\/compliant Intel), net\+C\+DF module, etc. and sets compiler environment variables.
\end{DoxyEnumerate}
\begin{DoxyItemize}
\item {\ttfamily source Theia\+\_\+setup} or
\item {\ttfamily source Yellowstone\+\_\+setup} or
\item {\ttfamily source Cheyenne\+\_\+setup}
\end{DoxyItemize}
\begin{DoxyEnumerate}
\item Make a build directory and change into it.
\end{DoxyEnumerate}
\begin{DoxyItemize}
\item {\ttfamily mkdir bin \&\& cd bin}
\end{DoxyItemize}
\begin{DoxyEnumerate}
\item Invoke cmake on the source code to build.
\end{DoxyEnumerate}
\begin{DoxyItemize}
\item {\ttfamily cmake ../src}
\end{DoxyItemize}
\begin{DoxyEnumerate}
\item Compile.
\end{DoxyEnumerate}
\begin{DoxyItemize}
\item {\ttfamily make}
\end{DoxyItemize}

\subsection*{Running the S\+CM with C\+C\+PP}


\begin{DoxyEnumerate}
\item First append the library path to the physics suite libraries that you need within the C\+C\+PP. For example, for the dummy schemes, use
\end{DoxyEnumerate}
\begin{DoxyItemize}
\item {\ttfamily setenv L\+D\+\_\+\+L\+I\+B\+R\+A\+R\+Y\+\_\+\+P\+A\+TH \$\+L\+D\+\_\+\+L\+I\+B\+R\+A\+R\+Y\+\_\+\+P\+A\+TH\textbackslash{}\+:\mbox{[}/path/to/build/directory/\mbox{]}ccpp/schemes/scm/src/scm-\/build}
\end{DoxyItemize}
\begin{DoxyEnumerate}
\item Run the S\+CM with one of the supplied cases (twpice, arm\+\_\+sgp\+\_\+summer\+\_\+1997, astex). The S\+CM will go through the time steps, applying forcing and calling the dummy physics.
\end{DoxyEnumerate}
\begin{DoxyItemize}
\item {\ttfamily ./gmtb\+\_\+scm twpice}
\end{DoxyItemize}

\subsection*{Setting up the physics suite}

First, a physics suite is defined using an X\+ML file located in src/ccpp/examples. The X\+ML file contains the suite name, the number of suite \char`\"{}parts\char`\"{} (suite parts exist so that an atmosphere \char`\"{}cap\char`\"{} can execute code between calls to the physics driver), the number of subcycles for each scheme (if physics schemes require smaller time steps than the dynamics), and the scheme names. Scheme names found in the suite X\+ML files must correspond to schemes located within the ccpp/schemes directory. Schemes within this directory are compiled as their own libraries, whose names must also be specified when using a scheme.

Using a suite in the S\+CM framework involves specifying its name in the case configuration file to be used. For example, for the twpice case (case\+\_\+config/twpice.\+nml), the variable \textquotesingle{}physics suite\textquotesingle{} is set to the desired suite name. N\+O\+TE\+: As mentioned in the \textquotesingle{}Running\textquotesingle{} section above, since the schemes are in their own libraries, you must specify the path to the compiled scheme libraries that are being used in the suite by appending the L\+D\+\_\+\+L\+I\+B\+R\+A\+R\+Y\+\_\+\+P\+A\+TH (or D\+Y\+L\+D\+\_\+\+L\+I\+B\+R\+A\+R\+Y\+\_\+\+P\+A\+TH for Mac OS). Without this step, the scheme libraries will not be found at runtime. This step is likely temporary and will be handled automatically at build time in future versions.

\subsection*{C\+C\+PP Implementation Notes}

Interaction with the C\+C\+PP is accomplished in two source files\+: gmtb\+\_\+scm.\+f90 and gmtb\+\_\+scm\+\_\+time\+\_\+integration.\+f90. Within gmtb\+\_\+scm.\+f90, before the time integration loop, the physics suite is initialized by calling \textquotesingle{}ccpp\+\_\+init\textquotesingle{} with the path to the suite X\+ML file and the ccpp\+\_\+t data type to be filled in. Immediately following, the \textquotesingle{}ccpp\+\_\+fields\+\_\+add\textquotesingle{} subroutine is called to fill in the ccpp\+\_\+t data type with the model data that is used (input, output, diagnostics, etc.) in the physics suite. Once the ccpp\+\_\+t data type is filled, the physics suite can be called with the \textquotesingle{}ccpp\+\_\+ipd\+\_\+run\textquotesingle{} routine, passing in the scheme to be called and the ccpp\+\_\+t data type. For the S\+CM, the first time step is handled after initialization. To advance in the time loop, the \textquotesingle{}do\+\_\+time\+\_\+step\textquotesingle{} subroutine is called from gmtb\+\_\+scm\+\_\+time\+\_\+integration.\+f90. This subroutine handles applying S\+CM forcing and calling the physics. For testing purposes, the parts, subcycle loops, and schemes within the physics suite are looped through, although calls to ccpp\+\_\+ipd\+\_\+run can be split apart as necessary with code executed between calls. The \textquotesingle{}atmosphere\textquotesingle{} cap has the control over how to call the schemes. 